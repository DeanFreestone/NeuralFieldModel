%% Tex file of coding notes of analytic implementation of neural
\documentclass[a4paper, 8pt, english]{article}

%% packages
\usepackage{amsmath}
\usepackage{graphicx}

%% title of this manuscript
\title{Matlab Code Notes for Neural Field Model}
\author{Miao Cao}
\date{1 July 2016}


\begin{document}

% title
\begin{titlepage}\centering
\vspace*{\fill}
\maketitle
\vspace*{\fill}
\end{titlepage}

\tableofcontents

\newpage
%% sections start from here


\section{Convolution of Two Gaussian basis functions}

\paragraph{Analytically prove convolution of two Gaussians. In general, the
precision level of analytic simulation should be around $10^{-6}$.}

\paragraph{File (Matlab script): Convolution2DGaussians.m}

\subsection*{Matlab script description}

Appendix E (Freestone et al., 2011, NeuroImage), Equation (4) $\left(\phi_{i}\otimes\phi_{j}\right)(r)=(\frac{\pi\sigma_{i}^{2}\sigma_{j}^{2}}{\sigma_{i}^{2}+\sigma_{i}^{2}})^{\frac{n}{2}}\exp(-\frac{1}{\sigma_{i}^{2}+\sigma_{i}^{2}}r^{T}r)$.

\subparagraph{Section: Generate spatial data, create a $NPoints\times NPoints$ cortical surface and phi and psi Gaussian basis functions.\\}


\begin{tabular}{|c|c|}
\hline
Variable name & Explanation\tabularnewline
\hline
X & Coordinates of x-axis\tabularnewline
\hline
Y & Coordinates of y-axis\tabularnewline
\hline
mu\_phi & Centre of gaussian\tabularnewline
\hline
sigma\_phi & covariance matrix of phi\tabularnewline
\hline
phi & Gaussian basis function\tabularnewline
\hline
mu\_psi & Centre of gaussian\tabularnewline
\hline
sigma\_psi & covariance matrix of psi\tabularnewline
\hline
psi & Gaussian basis function\tabularnewline
\hline
\end{tabular}

\paragraph{Analytic check of convolution of two Gaussians.\\}
\label{subp:Analytic check of convolution of two Gaussians}

%\subparagraph{Section: analytic check of convolution of two gaussians}

\begin{tabular}{|c|c|}
\hline
Variable name & Explanation\tabularnewline
\hline
mu & Centre of a gaussian resulted from convolution of two gaussians, phi
and psi\tabularnewline
\hline
var\_phi & variance of gaussian phi\tabularnewline
\hline
var\_psi & variance of gaussian psi\tabularnewline
\hline
exponential & exponential part of Equation Appendix E.4 (Freestone et al. 2011)\tabularnewline
\hline
CovMat & covariance matrix of a resultant gaussian convolved by two other gaussians\tabularnewline
\hline
coefficient & coefficient part of Equation Appendix E.4 (Freestone et al. 2011)\tabularnewline
\hline
convE2\_equivalent & analytic result of convolution of two gaussians, phi and psi\tabularnewline
\hline
\end{tabular}

\section{Compute Gamma}

\paragraph{File (Matlab script): ComputeGamma.m}

\subsection*{Matlab script description}

\paragraph{This matlab script mainly implement and estimate the following two
equations, (21) and Appendix (D.7) (Freestone et al. 2011 NeuroImage).}

\paragraph{Field, v(r), is decomposed of a finite-dimensional state vector and
a vector of Gaussian basis functions.}

\paragraph{Define $\Gamma=\int_{\Omega}\phi(r)\phi^{T}(r)dr$. Firstly, programmatically
define $\phi(r)$, as a vector of Gaussian basis functions. Each Gaussian
function is defined as $\phi(r-r')=\exp{(-\frac{(r-r')^{T}(r-r')}{\sigma_{\phi}^{2}})}$.}

\section{Compute Psi}
\label{compute psi}
\paragraph{Compute Psi --- connectivity kernel}
\label{par:Compute}

\section{Compute State Vector}
\label{state vector}
\paragraph{parameter table, figures}


\section{Compute State Vector --- Check with numerical solution}
\label{sec:Compute State Vector --- Check}
\section{In order to check the computation of state vectors, we firstly initialise a random field. }




\end{document}
