%% LyX 2.2.0 created this file.  For more info, see http://www.lyx.org/.
%% Do not edit unless you really know what you are doing.
\documentclass[english]{article}
\usepackage[T1]{fontenc}
\usepackage[latin9]{inputenc}
\usepackage{amsmath}

\makeatletter

%%%%%%%%%%%%%%%%%%%%%%%%%%%%%% LyX specific LaTeX commands.
%% Because html converters don't know tabularnewline
\providecommand{\tabularnewline}{\\}

\makeatother

\usepackage{babel}
\begin{document}

\title{Neural Field Model}
\maketitle

\section{Convolution of Two Gaussian basis functions}

\paragraph{Analytically prove convolution of two Gaussians. In general, the
precision level of analytic simulation should be around$10^{-6}$.}

\paragraph{File (Matlab script): Convolution2DGaussians.m}

\subsection*{Matlab script description}

\paragraph{Appendix E.4 (Freestone et al. 2011 NeuroImage) $\left(\phi_{i}\otimes\phi_{j}\right)(r)=(\frac{\pi\sigma_{i}^{2}\sigma_{j}^{2}}{\sigma_{i}^{2}+\sigma_{i}^{2}})^{\frac{n}{2}}exp(-\frac{1}{\sigma_{i}^{2}+\sigma_{i}^{2}}r^{T}r)$.}

\subparagraph{Section: Generate data, create a$NPoints\times NPoints$ cortical
surface and phi and psi Gaussian basis functions.}

\subparagraph{%
\begin{tabular}{|c|c|}
\hline 
Variable name & Explanation\tabularnewline
\hline 
X & Coordinates of x-axis\tabularnewline
\hline 
Y & Coordinates of y-axis\tabularnewline
\hline 
mu\_phi & Centre of gaussian\tabularnewline
\hline 
sigma\_phi & covariance matrix of phi\tabularnewline
\hline 
phi & Gaussian basis function\tabularnewline
\hline 
mu\_psi & Centre of gaussian\tabularnewline
\hline 
sigma\_psi & covariance matrix of psi\tabularnewline
\hline 
psi & Gaussian basis function\tabularnewline
\hline 
\end{tabular}}

\subparagraph{Section: analytic check of convolution of two gaussians}

\subparagraph{%
\begin{tabular}{|c|c|}
\hline 
Variable name & Explanation\tabularnewline
\hline 
mu & Centre of a gaussian resulted from convolution of two gaussians, phi
and psi\tabularnewline
\hline 
var\_phi & variance of gaussian phi\tabularnewline
\hline 
var\_psi & variance of gaussian psi\tabularnewline
\hline 
exponential & exponential part of Equation Appendix E.4 (Freestone et al. 2011)\tabularnewline
\hline 
CovMat & covariance matrix of a resultant gaussian convolved by two other gaussians\tabularnewline
\hline 
coefficient & coefficient part of Equation Appendix E.4 (Freestone et al. 2011)\tabularnewline
\hline 
convE2\_equivalent & analytic result of convolution of two gaussians, phi and psi\tabularnewline
\hline 
\end{tabular}}

\section{Compute Gamma}

\paragraph{File (Matlab script): ComputeGamma.m}

\subsection*{Matlab script description}

\paragraph{This matlab script mainly implement and estimate the following two
equations, (21) and Appendix (D.7) (Freestone et al. 2011 NeuroImage).}

\paragraph{Field, v(r), is decomposed of a finite-dimensional state vector and
a vector of Gaussian basis functions.}

\paragraph{Define $\Gamma=\int_{\Omega}\phi(r)\phi^{T}(r)dr$. Firstly, programmatically
define $\phi(r)$, as a vector of Gaussian basis functions. Each Gaussian
function is defined as $\phi(r-r')=e^{(-\frac{(r-r')^{T}(r-r')}{\sigma_{\phi}^{2}})}$.}
\end{document}
